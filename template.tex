\documentclass[10pt]{article}

% landscape, twocolumn
\usepackage[a4paper, landscape, twocolumn, twoside]{geometry}
% \usepackage[a4paper, top = 0.6in, bottom = 0.3in, left = 0.5in, right = 0.5in, landscape, twocolumn, twoside]{geometry}

\usepackage{calc}
\setlength{\topmargin}{-1in + 15pt}
\setlength{\headheight}{12pt}
\setlength{\headsep}{10pt}
\setlength{\footskip}{0pt}
\setlength{\textheight}{\paperheight - 60pt}

\setlength{\oddsidemargin}{-1in + 30pt}
\setlength{\evensidemargin}{-1in + 30pt}
\setlength{\textwidth}{\paperwidth - 60pt}

\usepackage[BoldFont, SlantFont, CJKchecksingle]{xeCJK}
\usepackage[xetex, CJKbookmarks, colorlinks]{hyperref}
\usepackage{fontspec, xunicode, xltxtra}
\usepackage{graphicx}
\usepackage{listings}
\usepackage{xcolor}
\usepackage{color}
\usepackage{amsmath}
\usepackage{amssymb}
\usepackage{fancyhdr}
\usepackage{zhfontcfg}
\setlength{\columnsep}{0.4in}

\pagestyle{fancy}
\fancyhead[LE,RO]{\bfseries\thepage}
\fancyhead[LO,RE]{\bfseries\leftmark}
\fancyhead[C]{\bfseries Doughnut◎Chiya}
\fancyfoot{}

% \newfontfamily{\lsttype}{Liberation Mono}

\definecolor{dkgreen}{RGB}{63, 127, 85}
\definecolor{dkpurple}{RGB}{127, 0, 85}
\definecolor{dkgrey}{RGB}{127, 127, 127}
\definecolor{dkred}{RGB}{154, 0, 0}
\lstset {
  basicstyle = \small\ttfamily,
  language = C++,
  aboveskip = 0pt,
  numbers = left,
  numberstyle = \footnotesize\ttfamily\color{dkgrey},
  numbersep = 5pt,
  tabsize = 2,
  breaklines = true,
  breakindent = 1.1em,
  keywordstyle = \color{dkpurple},
  commentstyle = \color{dkgreen},
  backgroundcolor = \color{white},
  stringstyle = \color{dkred},
  deletekeywords = {in},
  showspaces = false,
  basewidth = {0.5em, 0.4em},
  frame = trbl,
  rulecolor = \color{dkgrey},
  showstringspaces = false,
  escapeinside = {<TeX>}{</TeX>}
}
\begin{document}
% \title{\LARGE Reference Document \& Code Library}
% \date{}
% \author{\textbf{Obscure}}
% \maketitle
\tableofcontents
\newpage

\section{数据结构}
\subsection{线段树}
\lstinputlisting{code/Data_structure/线段树.cpp}
\subsection{树状数组}
\lstinputlisting{code/Data_structure/树状数组.cpp}
\subsection{并查集}
\lstinputlisting[language=C]{code/Data_structure/并查集.c}
\subsection{树链剖分}
\lstinputlisting{code/Data_structure/树链剖分.txt}
\lstinputlisting{code/Data_structure/树链剖分.pas}
\subsection{树的分治}
\lstinputlisting{code/Data_structure/树的分治.txt}
\lstinputlisting{code/Data_structure/树的分治.pas}
\subsection{可持久化线段树}
\lstinputlisting{code/Data_structure/可持久化线段树.txt}
\lstinputlisting{code/Data_structure/可持久化线段树.pas}
\subsection{可并堆}
\lstinputlisting{code/Data_structure/可并堆.txt}
\subsubsection{随机堆}
\lstinputlisting{code/Data_structure/随机堆.cpp}%nc
\subsubsection{左偏树}
\lstinputlisting{code/Data_structure/左偏树.pas}
\subsection{平衡树}
\lstinputlisting{code/Data_structure/平衡树.txt}
\lstinputlisting{code/Data_structure/平衡树.pas}
\lstinputlisting{code/Data_structure/sequence.pas}

\section{图论}
\subsection{2set}
\lstinputlisting{code/Graph/2set.txt}
\lstinputlisting{code/Graph/2set.cpp}
\subsection{欧拉路}
\lstinputlisting{code/Graph/欧拉路.cpp}  %nc
\subsection{LCA}
\subsubsection{倍增}
\lstinputlisting{code/Graph/LCA-倍增.cpp}
\lstinputlisting{code/Graph/LCA-倍增-bfs.pas}
\subsubsection{tarjan}
\lstinputlisting{code/Graph/LCA-tarjan.cpp}
\subsection{RMQ}
\subsubsection{ST}
\lstinputlisting{code/Graph/RMQ-ST.pas}
\subsubsection{线性}
\lstinputlisting{code/Graph/RMQ-线性.pas}
\subsection{最短路}
\subsubsection{SPFA}
\lstinputlisting{code/Graph/SPFA.cpp}
\subsubsection{dijkstra}
\lstinputlisting{code/Graph/dijkstra.cpp}
\subsection{最小生成树}
\subsubsection{prim}
%\lstinputlisting{code/Graph/prim.pas}
\subsubsection{kruskal}
\lstinputlisting{code/Graph/kruskal.cpp}
\subsection{最小树形图}
\lstinputlisting{code/Graph/最小树形图.txt}
\lstinputlisting{code/Graph/最小树形图.cpp}
\subsection{有向图强连通分量}
\lstinputlisting{code/Graph/tarjan-递归.cpp}
\lstinputlisting{code/Graph/tarjan-非递归.pas}
\subsection{哈密顿回路}
\lstinputlisting{code/Graph/哈密顿回路.txt}
\lstinputlisting{code/Graph/哈密顿回路.pas}
\subsection{关键路径}
\lstinputlisting{code/Graph/关键路径.pas}
\subsection{割点割边}
\lstinputlisting{code/Graph/割点.cpp}
\subsection{二分图匹配}
\lstinputlisting{code/Graph/匈牙利.cpp}  %nc
\subsection{网络流}
\subsubsection{上下界}
\lstinputlisting{code/Graph/上下界.txt}
\lstinputlisting{code/Graph/上下界.cpp}  %nc
\subsubsection{平面图最小割}
\lstinputlisting{code/Graph/平面图最小割.txt}
\subsubsection{费用流}
\lstinputlisting{code/Graph/费用流.cpp}  %nc
\subsubsection{最小路径覆盖}
\lstinputlisting{code/Graph/最小路径覆盖.cpp}%nc
\subsubsection{tips}
\lstinputlisting{code/Graph/tips.txt}

\section{数学}
\subsection{快速幂}
\lstinputlisting{code/Math/快速幂.cpp}
\subsection{线性筛法}
\lstinputlisting{code/Math/线性筛法.cpp}
\subsection{容斥}
\lstinputlisting{code/Math/容斥.cpp}     %nc
\subsection{拓扑}
\lstinputlisting{code/Math/拓扑.cpp}     %nc
\subsection{斜率优化}
\lstinputlisting{code/Math/斜率优化.txt}
\subsection{polya}
\lstinputlisting{code/Math/polya.txt}
\subsection{初等数论}
\lstinputlisting{code/Math/初等数论-simple.cpp}%nc

\section{字符串}
\subsection{KMP}
\lstinputlisting{code/String/KMP.cpp}
\subsection{exKMP}
\lstinputlisting{code/String/exKMP.cpp}
\subsection{AC自动机}
\lstinputlisting{code/String/AC自动机.pas}
\subsection{后缀数组}
\lstinputlisting{code/String/后缀数组.pas}
\subsection{后缀自动机}
\lstinputlisting{code/String/sam.cpp}

\section{计算几何}
\subsection{计算几何}
\lstinputlisting{code/Geometry/计算几何.txt}
\subsection{最远点对}
\lstinputlisting{code/Geometry/最远点对.cpp}
\subsection{半平面交}
\lstinputlisting{code/Geometry/半平面交.cpp}%nc

\section{DP}
\subsection{多重背包队列优化}
\lstinputlisting{code/DP/多重背包队列优化.cpp}%nc
\subsection{LIS}
\lstinputlisting{code/DP/LIS.txt}
\subsection{树型动规}
\lstinputlisting{code/DP/树型动规.txt}
\subsection{动规优化}
\lstinputlisting{code/DP/动规优化.txt}

\section{Others}
\subsection{高斯消元}
\lstinputlisting{code/Others/高斯xor.txt}
\subsubsection{高斯消元}
\lstinputlisting{code/Others/gauss-square.cpp}%nc
\subsubsection{xor方程组}
\lstinputlisting{code/Others/xor.pas}
\subsection{博弈论}
\lstinputlisting{code/Others/博弈论.txt}
\subsection{陈丹琦分治}
\lstinputlisting{code/Others/陈丹琦分治.txt}
\subsection{矩阵乘法}
\lstinputlisting{code/Others/矩阵乘法.txt}
\subsection{求AmoB+…nAmoB}
\lstinputlisting{code/Others/求AmoB+…nAmoB.pas}
\subsection{高精度}
\lstinputlisting{code/Others/高精.cpp}
\subsection{头文件}
\lstinputlisting{code/Others/头文件.cpp}

\section{Tips}
\subsection{对拍}
\lstinputlisting{code/Others/对拍.txt}
\subsection{class-map}
\lstinputlisting{code/Tips/class-map.cpp}
\subsection{FastIo}
\lstinputlisting{code/Tips/FastIo.cpp}
\subsection{JavaFastIo}
\lstinputlisting[language=Java]{code/Tips/JavaFastIo.java}
\subsection{javaSample}
\lstinputlisting[language=Java]{code/Tips/javaSample.java}
\subsection{tips}
\lstinputlisting{code/Tips/tips.cpp}

\end{document}
